\documentclass[11pt,a4paper,twoside]{book}
\usepackage{fontenc}[T1]
\usepackage[utf8]{inputenc}
\usepackage[english]{babel}
\usepackage{amssymb,amsmath}
\usepackage[top=2cm,bottom=2cm,left=2cm,right=2cm]{geometry}
\usepackage[pdftex]{graphicx} %per poter inserire le figure
\usepackage{amssymb,amsmath,amsthm,amsfonts}
\usepackage{xspace}
\usepackage{tabularx}
\usepackage{indentfirst}
\usepackage{subfigure}
\usepackage[small]{caption}
\usepackage{eucal}
\usepackage{eso-pic}
\usepackage{url}
\usepackage{booktabs}
\usepackage{afterpage}
\usepackage{parskip}
\usepackage{listings}
\usepackage{fancyhdr}
\usepackage{textcomp}
\usepackage{cite}
\usepackage{multirow}
\usepackage[utf8]{inputenc}   %per riuscire a scrivere gli accenti
\usepackage{setspace}
\usepackage{fancyhdr}
\newenvironment{abstract}{\cleardoublepage\thispagestyle{empty}\null\vfill\begin{center}\bfseries\abstractname\end{center}}{\vfill\null}
\pagestyle{fancy} 


\begin{document}
	\frontmatter
	\begin{titlepage}
		\vspace{5mm}
		\begin{figure}[hbtp]
			\centering
			\includegraphics[scale=.16]{Logo_unipd.png}
		\end{figure}
		\vspace{5mm}
		\begin{center}
			{{\huge{\textsc{\bf UNIVERSIT\`A DEGLI STUDI DI PADOVA}}}\\}
			\vspace{5mm}
			{\Large{\bf Dipartimento di Fisica e Astronomia ``Galileo Galilei''}} \\
			\vspace{5mm}
			{\Large{\textsc{\bf Master Degree in Physics}}}\\
			\vspace{20mm}
			{\Large{\textsc{\bf Final Dissertation}}}\\
			\vspace{30mm}
			\begin{spacing}{3}
				{\LARGE \textbf{Proibing the Reheating Phase After Inflation}}\\
			\end{spacing}
			\vspace{8mm}
		\end{center}
		
		\vspace{20mm}
		\begin{spacing}{2}
			\begin{tabular}{ l  c  c c c  cc c c c c  l }
				{\Large{\bf Thesis supervisor}} &&&&&&&&&&& {\Large{\bf Candidate}}\\
				{\Large{\bf Prof. Nicola Bartolo}} &&&&&&&&&&& {\Large{\bf Raffaele Giusti}}\\
				{\Large{\bf Thesis co-supervisor}}\\
				{\Large{\bf Prof./Dr. Name Surname}}\\
			\end{tabular}
		\end{spacing}
		\vspace{15 mm}
		
		\begin{center}
			{\Large{\bf Academic Year 2021/2022}}
		\end{center}
	\end{titlepage}
	\clearpage{\pagestyle{empty}\cleardoublepage}

\begin{abstract}
	Inflation is the standard scenario, completely consistent with a variety of data, to understand the  
	generation of primordial scalar (density) perturbations, i.e.  the seeds of all the cosmological structures we  
	see today, and also of tensor perturbations (i.e. primordial gravitational waves). Inflation must come to an  
	end, in order for the universe to be filled in with radiation, and to proceed  through the standard radiation  
	dominated era, during which, e.g. primordial nucleosynthesis can take place. Such a transition (called  
	reheating phase) from the inflationary stage to the standard radiation dominated epoch, is the least known  
	part of the inflationary scenario, because, e.g., it involves couplings of the fields driving inflation to other  
	(relativistic) particles.  
	
	Nonetheless the precision of cosmological data has allowed recently to put already some constraints on  
	such a reheating phase.  
	
	This Thesis will provide an up-to-date review of all the cosmological observables that can be used to open  
	a new window into such period of the early universe. Moreover it will focus on an alternative observational  
	probe that has been investigated recently, namely spatial anisotopies of the energy density of primordial  
	gravitational waves from inflation. The Thesis will explore how such anisotropies depend on, e.g., the  
	equation of state of the Universe during the reheating phase, or e.g., on the number of relativistic degrees  
	of freedom generated during this phase and their thermalisation properties.
\end{abstract}
	
\tableofcontents
	

\chapter{Introduction}

In modern cosmology one of the most important theories is represented by \textit{cosmological inflation}. \\
Inflation was an era during the early hystory of the Universe, before the epoch of primordial nucleosynthesis, during which the Universe expansion was accelerated. Such a period can be attained if the energy density of the Universe is dominated by the vacuum energy density associated with the potential of a scalar field, called the \textit{inflaton field} $ \phi $. \\
Inflation leads to a very rapid expansion of the Universe and can elegantly solve the flatness, the horizon and the monopole problems of the Standard Big Bang Cosmology (indeed, the first model of inflation by Guth in 1981 was introduced to address such problems \cite{Guth:Intro}). It can explain the production of the first density perturbations in the early Universe which are the seeds for the Large Scale Structure (LSS) in the distribution of galaxies and underlying dark matter, and for the Cosmic Microwave Background (CMB) temperature anisotropies that we observe today. Inflation has become the dominant paradigm to understand the initial conditions for structure formation and CMB anisotropies.\\
During this era primordial density fluctuations and gravitational waves are created from quantum fluctuations \textquotedblleft redshifted" out of the horizon during the rapid expansion and here \textquotedblleft frozen\textquotedblright.\\
From this period we can observe temperature anisotropy in the CMB caused by perturbations at the surface of the last scattering. The CMB temperature anisotropies are first detected by the Cosmic Background Explorer (COBE) satellite \cite{COBE1:intro},\cite{COBE2:intro}. \\
Another impressive confirmation of the inflationary theory has been provided by the data of the Wilkinson Microwave Anisotropy Probe (WMAP) mission that has produced a full-sky map of the angular variations of the CMB with unprecedented accuracy. WMAP data confirm the inflationary mechanism as responsible for the generation of superhorizon fluctuations \cite{WMAP:intro}, \cite{NonGauss:Intro}.\\
More recently, the best constraints on the CMB data are provided by the \textit{2018 Planck measurements} \cite{Planck2018:intro}. The \textit{Planck} data have given a very precise characteritation of the primordial cosmological perturbations and have allowed cosmological parameters to be constrained at the sub-percent level. Thus, \textit{Planck} measurements provide  a powerful constraint to inflationary models.\\

The literature contains a large number of different models of inflation. Each model amounts to a choice for the potential  of the inflaton, plus a period of ending inflation, called \textit{reheating}.\\
The main models of inflation focuse on two different paradigms. Throughout the early 1990s discussion was dominated by the \textit{single-field models}. In these models, the scalar field potential often is chosen to be some convenient simple function, such as a monomial or exponential, and the initial conditions are chosen such that the scalar field is well displaced from any minimun.\\
In the mid-1990s, this paradigm was challenged by a new wave of inflationary model building, based on particle physics motivation such as the theories of supersymmetry, supergravity, and superstrings. In such a period we have a new class of models, known generically as \textit{hybrid inflation}, which rely on interactions between two scalar fields and utilize the flat potentials expected in supersymmetry theories \cite{Liddle:intro}. \\
In other scenarios we have the presence of a further scalar field besides the inflation that does not influence the inflationary dynamics (for example the \textit{curvaton scenario}), or the inflaton coupled to a scalar field or a gauge field. Finally, there are theories based on Modified Gravity (MG) that involve a modification of General Relativity \cite{GWFromInflation:Intro}.\\

Inflation cannot proceed forever. In fact, the greatest successes of the Standard Big Bang model, such as primordial nucleosynthesis and the origin of the CMB, require the standard evolution from radiation to matter dominated era.\\ 
The transition from inflation to later stages of the evolution of the Universe (radiation and matter dominance) is referred to as \textit{reheating}. In the simplest models (single-field, slow-roll scenario) inflation ends when the inflaton field starts rolling fast along its potential, it reaches the minimun and then oscilates around it. During reheating the inflaton loses its energy, eventually leading the production of ordinary matter.\\
 More intricate scenarios include non perturbative processes such as (broad) resonance decay, tachyonic instability, instant preheating, fermionic preheating. The \textit{preheating} denotes the initial stage of reheating where we have an exponentially decay that generate high occupation numbers in selected frequency bands.  \\ 
 The aim of this Thesis is provide a complete and up-to-date review of the main models of reheating in the literature.
However, the reheating era is difficult to constrain observationally. In the absence of topological defects like monopoles or strings, the fluctuations produced during reheating remain sub-horizon and cannot leave an observable imprint at the level of the CMB or LSS. A lower bound is placed on the reheating temperature (\textit{i.e}. the temperature at which the standard radiation era of the Universe begins after reheating) by primordial nucleosynthesis (BBN) $ T_{BBN} \sim  10^{-2} Gev $ \cite{Steigman:nucleosynthesisIntro}. The scale of inflation is bounded from above and can be as large as $ \sim 10^{16} Gev $, leaving for the reheating temperature an allowed range of many orders of magnitude. 
Moreover, a variety of signatures relative to production of primordial black holes, magnetic field, unwanted relics, and also to mechanisms such as baryo- and leptogenesis, may be traced back to specific preheating/reheating models \cite{ReheatingPredictionsSingleFieldModel:intro} .\\ 

An extremely important prediction of cosmological inflation is the generation of a Stochastic Background of primordial Gravitational Waves (SGWB). Primordial GW are in fact not expected in the non inflationary standard early-universe models and will provide, if detected, a \textit{smoking gun} probe of inflation. In the standard slow roll inflationary scenario tensor fluctuations of the metric (\textit{i.e} primordial GW) are characterized by a nearly scale invariant spectrum on super-horizon scales. The amplitude of the GW signal is usually described by the tensor-to-scalar ratio \textit{r}, defined as the ratio between the tensor and scalar power spectrum amplitudes at a given pivot scale $ k_{*} $.\\
The current best bound on $ \textit{r} $ comes from the joint analysis of Planck, BICEP2, Keck Array and other data, which yields $ r < 0.07 $ at $ 95 \% $ C.L. for $ k_{*}$ = $ 0.05\ Mpc^{-1} $ \cite{Bicep2:Intro}. \\
A crucial point is that, even in the simplest single field framework, different inflationary scenarios predict different values of r. The study of observational signatures of primordial GW thus provide not only a way to probe the general inflationary theory, but also to discriminate in detail among specific models.\\
A detection of primordial GW would not only be extremely important for Cosmology but also for High Energy and Fundamental Physics. Since the energy scale of inflation is directly linked to the value of the tensor-to-scalar ratio, by means of a detection of $ r $ we would obtain a hint of the the physics beyond the Standard Model and the precise indication of the energy regime of such new physics.\\
Indeed, the primordial GW are the object of a growing experimental effort, and their detection will be a major goal for Cosmology in the forthcoming  decades. \\
The main observational signature of the inflationary GW background is a curl-like pattern (B-mode) in the CMB. A number of, present or forthcoming, ground-based or baloon-borne experiments, are specifically aimed to B-mode detection. Unfortunately, the B-modes measured by BICEP2 \cite{Bicep2BMode:Intro} did not point to any inflationary signal, but several next-generation CMB space missions have been proposed in recent years with the specific goal of B-mode detection such as COrE \cite{COre:intro} or PRISM \cite{PRISM:intro}. 
Finally, we have the possibility of a future direct detection, by experiments such as aLIGO \cite{LIGO:intro}, or eLISA \cite{Lisa:Intro}, especially if some inflationary models produced a blue-tilted primordial tensor spectrum \cite{GWFromInflation:Intro}.\\
In this Thesis we will focus on another type of GW: those generated by classical mechanism during reheating after inflation. By investigating  GW we cannot neglect this stage. In fact, there are many models for the reheating period which provide further GW production, besides that inflationary stage. Moreover, it can be shown that reheating parameters are related to inflationary power-spectra ones, so that the constraints on tensor perturbations are related to those on the reheating period.


In this work we are going to review all the most important inflation and reheating scenarios. After the first two chapters, we will present detailed models of reheating about each of its stages (Preheating, Bubbly Stage, Scalar Wave Turbolence,Thermalitation) investigated in the literature. Moreover, we will review predictions about the production of GW, observable signatures on CMB and the possibility of direct detection of GW from reheating epoch.\\
The Thesis is organized with the following plan:
In the chapter 1 we review the standard single-field model of slow-roll inflation and we derive the physical observables that can be seen and constrained from CMB. \\
In chapter 2 we overview the main inflationary models that are important for the subsequent reheating epoch. Moreover, we present the principal signatures on CMB coming from GWs and reheating parameters. We end this chapter describing a toy model of reheating.\\
In the chapter 3, we investigate about \textit{preheating}, the first stage of reheating, in the model of parametric resonance, and we examinate the GWs generated in this model and observable today. \\
In  chapter 4 we examinate the other important model of preheating: preheating after hybrid inflation. Then, we briefly consider other models of preheating such as fermionic preheating, instant preheating and multi-field reheating.\\
In  chapter 5 and 6 we examinate the subsequent and final stages after preheating: the bubbly stage and thermalitation.\\
In chapter 7 we  overview the principal signatures  of the reheating epoch and we present a new approach to study the reheating epoch that consists in examinating the spatial anisotropies of the energy density of primordial gravitational waves from inflation. 
 
 \mainmatter
 
\chapter{Inflation}

The inflation theory not only provides an excellent way to solve flatness and horizon problems but also generates density perturbations as seeds for large-scale structure in the universe. Quantum fluctuations of the field that drives inflation (called \textit{inflaton}) are streched on large scales by the accelerated expansion. In the simplest version of the single-field scenario the fluctuations are \textquotedblleft frozen" after the scale of perturbations leaves the Hubble radius during inflation. Long after the inflation ends, the perturbations cross inside the Hubble radius again. Thus, inflation provides a causal mechanism for the origin of large-scale structure in the universe. An important prediction of inflation is that density perturbations generally exhibit nearly scale-invariant spectra. This prediction can be directly tested by the measurement of the temperature anisotropies in Cosmic Microwave Background (CMB). Indeed, the anisotropies observed by the Cosmic Background Explorer (COBE) in 1992 showed nearly scale invariant spectra. All data acquired until now have continued to confirm the main predictions of the inflationary theory within observational errors \cite{InflationDynamicsAndReheating:chap1}.\\
In this first chapter we review the simple standard model of slow-roll inflation and we derive observables and relations that will be used in the next chapters.

\section{Standard big-bang cosmology}

The standard big-bang cosmology is based upon the cosmological principle which requires that the universe is homogeneous and isotropic on averaging over large volumes.\\
A homogeneous and isotropic universe is described by the Friedmann-Robertson-Walker metric (FRW):

\begin{equation}
	\label{metric}	
	ds^{2}   = - dt^{2} + a^{2}(t)\Big[\frac{dr^{2}}{1-Kr^{2}}  +  r^{2}(d\theta^{2} + sin^{2} \theta d\phi^{2})\Big] . 
\end{equation}
Here $a(t)$ is the scale factor with $ t $ being the cosmic time and $(r,\theta,\phi) $ are comoving (spherical) coordinates. The constant $ K $ is the spatial curvature, where positive, zero, and negative values correspond to closed, flat, and hyperbolic spatial sections respectevely. \\
The evolution of the universe is dependent on the material within it. A key role is played by the \textit{equation of state} relating the energy density $ \rho (t) $ and the pressure $ P(t) $. Assuming a perfect and barotropic fluid we can describe it with the relation $ P=\omega\rho $ with $ \omega=0 $ for non-relativistic matter (dust) and $ \omega=1/3 $ for radiation.
For example in the universe we have $ P=0 $ or $ P=1/3 $ if it is dominated by dust or radiation, respectevely. \\
The dynamical evolution of the universe is known once we solve the Einstein equations for General Relativity:

\begin{equation}
	\label{eq:gr}
	G_{\mu\nu}  = R_{\mu\nu} - \frac{1}{2}g_{\mu\nu}R=8\pi G T_{\mu\nu}
\end{equation}

where $ G_{\mu\nu} $ is the Einstein Tensor, $ R_{\mu\nu} $, $ R $, $ T_{\mu\nu}$, and G are the Ricci tensor, Ricci scalar, energy-momentum tensor and gravitational constant, respectively. The energy momentum tensor $ T_{\mu\nu} $ describes the perfect fluid with density energy $ \rho $ and isotropic pressure P that fills the universe.\\
The Planck energy, $m_{pl}=1.2211 \times 10^{19}$ GeV, is related to G through the relation $ m_{pl} = (\hslash c^{5}/G)^{1/2}$. Hereafter we use the units $ \hslash = c = 1 $.\\
From the Einstein equations (\ref{eq:gr}) for the background FRW metric (\ref{metric}), we obtain the Friedmann Equations:
\begin{equation}
	\label{friedmannEquations1}
	H^{2}=\frac{8\pi G}{3}\rho - \frac{K}{a^{2}},
\end{equation}
\begin{equation}
	\label{friedmannEquations2}	
	\frac{\ddot{a}}{a} = -\frac{4\pi G}{3}(\rho + 3P),
\end{equation}

where the dots denote the derivative with respect to t, and $ H=\dot{a}/a $ is the Hubble expansion rate. Combining these relations we obtain the energy conservation equation 

\begin{equation}
	\label{energyCons}
	\dot{\rho} + 3H(\rho + P)=0,
\end{equation}

which is known as the continuity or fluid equation.\\
The Friedmann equation (\ref{friedmannEquations1}) can be rewritten as 

\begin{equation}
	\label{densityParameter}
	\Omega - 1 = \frac{K}{a^{2}H^{2}}
\end{equation}

where 

\begin{equation}
	\label{criticalDensity}
	\Omega=\frac{\rho}{\rho_{c}}, \quad    with \quad   \rho_{c}=\frac{3H^{2}}{8\pi G}
\end{equation}

where the density parameter $ \Omega $ is the ratio of the energy density to the critical density $ \rho_{c} $.\\
When the spatial geometry is flat (K=0, $ \Omega $=1) the solution for equations (\ref{friedmannEquations1})
and (\ref{friedmannEquations2}) are $ a \propto t^{1/2} \quad  (\rho \propto a^{-4}) $ for the radiation dominated era, and $ a\propto t^{2/3} \quad  (  \rho \propto  a^{-3}$) for dust dominated era.\\
Thus, from these equations we obtain a decelerate expansion ($ \ddot{a} < 0 $) for the universe.\\

\section{Standard cosmology and inflation}
In this section we briefly review the problems with the standard cosmology and how are solved by the idea of inflation \cite{Liddle:intro},\cite{Dodelson:Chap1}, \cite{InflationDynamicsAndReheating:chap1}. 

\subsection*{\emph{Flatness problem}}
In the standard big-bang theory with $ \ddot{a} < 0  $, the $ a^{2}H^{2} (= \dot{a}^{2}) $ term in (\ref{densityParameter}) always deacreases. This means that
$ \Omega $ tends to evolve away from unity with the expansion of the universe. Howewer, since present observations suggest that $ \Omega  $ is within a few percent of unity today, $ \Omega $ is forced to be much closer to unity in the past. For example, we require $ |\Omega-1| < \mathcal{O}(10^{-16}) $ at the epoch of nucleosynthesis and  $ |\Omega-1| < \mathcal{O}(10^{-64}) $ at the Planck epoch \cite{Liddle:intro}. This appears an extreme and innatural fine-tuning of initial conditions. Unless initial conditions are chosen  very accurately, the universe either collapses too soon, or expands too quickly before the structure can be formed. This is the so-called \textit{flatness problem}.

\subsection*{\emph{Horizon problem}}
Consider a comoving wavelength, $ \lambda $, and the corresponding physical wavelength, $ a\lambda $, which at some time is inside the Hubble radius, $ H^{-1} $ (\emph{i.e. a$ \lambda \leq  H^{-1}  $}). The standard big-bang cosmology is characterized by the cosmic evolution of  $a \propto t^{n} $ with 0 $< n < 1$. In this case the physical wavelength grows as $ a\lambda \propto t^{n} $, whereas the Hubble radius evolves as $H^{-1}\propto t$ . Therefore the physical wavelength becomes much smaller than the Hubble radius at late times.
This means that a causally connected region can only be a small fraction of the Hubble radius.\\
For example if we observe photons in the cosmic microwave background (CMB) which are last-scattered at the time of decoupling turns out that the causally connected regions on the surface of last scattering corresponds to an angle of order 1°.\\
This appears to be in contrast with observations of the CMB which has the same temperature to high precision in all directions on the sky. There is no way to establish thermal equilibrium if these points were never been in causal contact before last scattering. This is the so-called \textit{horizon problem} .

\subsection*{\emph{Large-scale structure}}
Experiments which observe temperature anisotropies in the CMB find that the amplitude of the anisotopies is small and their power spectrum is close to scale-invariant on large scales. It is impossible to generate such fluctuations  via causal processes in a FRW metric in the time between the big bang and the time of the last scattering.

\subsection*{\emph{Relic density problem}}
In Particle Physics the standard paradigm to study the fundamental interactions is the Spontaneous Symmetry Breaking (SSB) of gauge symmetries.\\
In the early universe the breaking of such symmetries leads to the production of many unwanted relics such as monopoles, cosmic strings, and other topological defects \cite{TopDefects:Linde}. For example, any grand unified theory based on a simple Lie group that includes the U(1) of electromagnetism must produce monopoles. String theories also predict supersymmetric particles such as gravitinos, Kaluza-Klein particles, and moduli fields.\\
If these massive particles exist in the early stages of the universe and they are stable (or sufficiently long-lived) could become the dominant matter in the early universe depending on their number density and therefore contradict a variety of observation such as those of the light element abundances. This problem is known as the \textit{relic density problem}.

\subsection*{\emph{The idea of inflation}}

The problems in the standard big bang cosmology lie in the fact that the universe always exhibits decelerated expansion. Instead, let us assume  the existence of a stage in the early universe with an accelerated expansion: 

\begin{equation}
	\label{acc.expansion}
		\ddot{a} > 0
\end{equation}

From the relation (\ref{friedmannEquations2}) this gives the condition

\begin{equation}
	\label{equationOfState}
	\rho + 3P < 0
\end{equation}

and, from  the equation of state $ P = \omega \rho  $, we obtain 

\begin{equation}
	\label{w parameter}
	\omega < -\frac{1}{3}
\end{equation}

The condition (\ref{acc.expansion}) essentialy means that $ \dot{a}\ (= aH) $ increases during inflation and hence that the comoving Hubble radius, $ (aH)^{-1} $, decreases in the inflationary phase.

\subsubsection*{\textit{Flatness problem}}
Since the $ a^{2}H^{2} $ term in (\ref{densityParameter}) increases during inflation, $ \Omega $ is rapidly driven towards unity. After the inflationary period ends, the evolution of the universe is followed by the conventional big-bang phase and $|\Omega-1| $ begins to increase again. As long as the inflationary expansion lasts sufficiently long and drives $ \Omega $ very close to one, $ \Omega $  will remain close to unity even in the present epoch.  

\subsubsection*{\textit{Horizon problem}}
Since the scale factor evolves approximately as $ a \propto t^{n} $ with n $ > 1 $ during inflation, the physical wavelength, $a \lambda $, grows faster than the Hubble radius, $ H^{-1}(\propto t)$. Therefore, physical wavelenghts are pushed outside the Hubble radius during inflation and then causally connected regions can be much larger than the Hubble radius, thus potentially solving the horizon problem.\\
A detailed computation shows this is achieved when the universe expands at least about $ e^{70} $ times during inflation, or 70 e-folds of expansion \cite{Liddle:intro}.

\subsubsection*{\textit{Large-scale structure}}
The inflationary period leads to perturbations of the scalar field that drives inflation and then of the energy density of the universe. In the early stage of inflation the scales of these perturbations are well within the Hubble radius and causal physics can works generating small quantum fluctuations. During the later stages these scales are pushed outside the Hubble radius (\textit{i.e.} the first Hubble radius crossing). Fluctuations of the scalar field become over-damped on long-wavelengths and the perturbations can be described as classical on these large scales. After the inflationary period these scales of perturbations  cross inside the Hubble radius again (\textit{i.e.} the second Hubble radius crosssing).\\
The small perturbations imprinted during inflation have amplitudes determined by the Hubble rate which is approximatevely constant during this period and hence  leads to an almost scale-invariant spectrum with constant amplitude on different scales.\\
In this way the inflation naturally provides a causal mechanism to generate the seeds of density perturbations observed in the CMB anisotropies.

\subsubsection*{\textit{Relic density problem}}
During the inflationary phase the energy density of massive particles scales as $ a^{-3} $, much faster than the energy density  of the universe (considering $ a \propto t^{n} $ with $ n>1 $, we have $ H \propto t^{-1} \propto a^{-1/n}$ that leads $  \rho \propto a^{-2/n} $). Thus, these particles are red-shifted away during inflation, solving the monopole problem.

\section{The inflaton equation}
As we have seen  from (\ref{w parameter}) a period of inflation is possible if the pressure $ P $ is negative with 

\begin{equation}
\label{Pressure-density}
P < -\frac{\rho}{3}
\end{equation}

In the special case in which $ \omega = -1 $ ($P=-\rho$) we have a period in the hystory of the universe called \textit{de Sitter stage}. \\
Such a period can be obtained inserting a cosmological constant $ \Lambda $ in the Einstein Equations: 

\begin{equation}
	\label{RGModified}
	R_{\mu\nu} - \frac{1}{2}g_{\mu\nu}R = 8\pi G T_{\mu\nu} - \Lambda g_{\mu\nu}.
\end{equation}

Considering the energy continuity equation (\ref{energyCons}) and the first Friedmann equation (\ref{friedmannEquations1}) we see that in a de Sitter phase $ \rho = constant $ and $ H = H_{inf} \simeq constant $ (we neglect the curvature K which is soon redshifted away as $ a^{-2} $).\\
Solving the second Friedmann equation (\ref{friedmannEquations2}) we obtain an exponentially growing of the scale factor

\begin{equation}
	\label{scaleFactorDeSitter}
	a(t)=a_{i} e^{ H_{inf}(t-t_{i})}	
\end{equation}

where $t_{i}$ is the time at which inflation starts. \\
The cosmological constant can be interpreted as the energy of the quantum vacuum state of the system, the \textit{vacuum} energy density contributed by any particle species. Even if it is simple to have such exponentially expansion of the universe the inflation must end at some point. Thus, there must be some dynamics regulating such system.\\
We can obtain this if we consider a scalar field, called the \textit{inflaton}, that dominates the energy in this epoch and leads the expansion.\\

Consider in full generality the total action

\begin{equation}
	\label{totalAction}
	S^{TOT} = S_{HE} + S_{\phi} + S_{m} 
		        = \frac{1}{16\pi G} \int d^{4} x \sqrt{-g} (R + \mathcal{L} [\phi , \partial_{\mu} \phi ] + \mathcal{L}_{matter}) 	
\end{equation}

where $ g $ is the determinant of the metric tensor $ g_{\mu\nu} $, $ S_{HE} $ is the Hilbert-Einstein action, $ S_{\phi} $ is the inflaton scalar field action and $ S_{m} $ is the action of the rest of the matter besides the inflaton (fermions, gauge fields, other scalars...). We will neglect $ S_{m} $ because, in general, they are subdominant at early times. \\
The action for a minimally-coupled real scalar field $ \phi $ is given by

\begin{equation}
	\label{actionScalarField}
	S=\int d^{4} x \sqrt{-g} \mathcal{L} = \Big [-\dfrac{1}{2} g^{\mu\nu} \partial_{\mu}\phi \partial_{\nu} \phi - V(\phi)\Big]	
\end{equation}

where $ V(\phi) $ specifies the scalar field potential and  can have different forms depending on the model. For example it can be a simple quadratic potential $ V(\phi)=\frac{1}{2}m^{2}\phi^{2} $, with $ m $ the mass of the particle associated to $ \phi $,$  $ or can describes  self interactions $ V(\phi) = \frac{\lambda}{4} \phi^{4} $. $ V(\phi) $ can represent also interactions of $ \phi $ with other fields and contains quantum radiative corrections. \\

To characterize the evolution of the scalar field in an expanding universe we can associate to $ \phi $ its stress-energy momentum $ T_{\mu\nu} $, which in General Relativity is given from a generic lagrangian by

\begin{equation}
	\label{stressEnergyMomentum}
	T_{\mu\nu}= \frac{-2}{\sqrt{-g}}\frac{\delta S}{\delta g^{\mu\nu}}=
	\frac{-2}{\sqrt{-g}}\Big [-\frac{\partial (\sqrt{-g} \mathcal{L})}{\partial g^{\mu\nu}} + \partial _{\alpha} \frac{\partial (\sqrt{-g} \mathcal{L})}{\partial \partial_{\alpha} g^{\mu\nu}} + ... \Big ].
\end{equation}

Assuming a minimally-coupled field (\textit{i.e.} doesn't involve direct coupling with gravity, e.g. $ \varepsilon  \phi^{2} R$ terms), the stress-energy tensor assumes the form

\begin{equation}
	\label{tensorScalrField2}
	T^{\phi}_{\mu\nu} = -2 \frac{\partial \mathcal{L}_{\phi}}{\partial g^{\mu\nu}} + g^{\mu\nu} \mathcal{L}_{\phi}
 =\partial_{\mu} \phi \partial_{\nu} \phi + g_{\mu\nu} \Big [-\frac{1}{2}g^{\mu\nu}\partial_{\mu}\phi\partial_{\nu}\phi - V(\phi)\Big]
\end{equation}

The equation of motion for $ \phi $ is derived by varying the action (\ref{actionScalarField}) with respect $ \phi $ obtaining the Klein-Gordon equation

\begin{equation}
	\label{KGEquation}
	\square  \phi = \frac{\partial V}{\partial \phi}	
\end{equation}

where $\square$ is the covariant D'Alembert operator

\begin{equation}
	\label{alembertOperator}
	\square \phi = \frac{1}{\sqrt{-g}}\partial_{\nu}\Big (\sqrt{-g} g^{\mu\nu} \partial_{\mu}\phi\Big ).
\end{equation}

In a FRW universe described by the metric (\ref{metric}), the evolution equation for $ \phi $ becomes

\begin{equation}
	\label{eomForPhi}	
	\ddot{\phi} + 3H\dot{\phi} - \frac{\nabla^{2} \phi}{a^{2}} + V'(\phi)=0
\end{equation}

where $ V'(\phi)=dV/d\phi $. Through $ 3H\dot{\phi} $ the field "feels" a friction due to the expansion of the universe, which will play a crucial role.\\

We can now express the inflaton field $ \phi $ as the sum of the classical background value and the field fluctuations

\begin{equation}
	\label{splitInflaton}
	\phi(t,\textbf{x})=\phi_{0}(t) + \delta \phi(t,\textbf{x})
\end{equation}

where $\phi_{0}(t) =\ <0|\phi(t,\textbf{x})|0> $  is the classical field, that is the expectation value of the inflaton field on the initial isotropic and homogeneous state, while $\delta \phi (t,\textbf{x})$ represents the quantum fluctuations around $ \phi_{0}(t)$.
We will consider first the background dynamics and then the evolution of quantum perturbations during inflation. This separation is justified by the fact that quantum fluctuations are much smaller than the classical value and therefore negligible when looking at the classical evolution.

\section{Classical Dynamics}
The inflaton field $ \phi(t) $  behaves like a perfect flud with background energy density and pressure given by 

\begin{equation}
	\label{energyDensityPressure}
	T^{0}_{0} = -\Big [\dfrac{1}{2} \dot{\phi}^{2}_{0}(t) + V(\phi_{0}) \Big ] = -\rho_{\phi}(t)
\end{equation}
\begin{equation*}
	T^{i}_{j} = -\Big [\dfrac{1}{2} \dot{\phi}^{2}_{0}(t) - V(\phi_{0}) \Big ]\delta^{i}_{j} = \delta^{i}_{j} P_{\phi} (t)	
\end{equation*}

where $ \rho_{\phi} (t) $ is the energy density and $ P_{\phi} (t) $ the isotropic pressure. $ T_{\mu\nu} $ is diagonal and the spatial part is the same in every direction as a consequence of isotropy and homogeneity resulting in a tensor typical of perfect fluids. Hereafter we don't insert the subscript \textquotedblleft 0" when we denote the background field. \\
Therefore if 

\begin{equation}
	\label{conditionFlatPotential}
	V(\phi) \gg \frac{1}{2} \dot{\phi}^{2}
\end{equation}

we see that 

\begin{equation}
	\label{deSitterCondition}
	P_{\phi} \simeq -\rho_{\phi}
\end{equation}

which gives rise to a quasi-de-Sitter phase.\\
From this simple calculation we realize that a scalar field whose energy is dominant in the universe and whose potential energy dominates over the kinetic term gives inflation. The condition (\ref{conditionFlatPotential}) is called \textit{slow-roll} regime, during which $ V(\phi) \simeq constant $ provides accelerated expansion driven by the vacuum energy density of $ \phi $, which mimics an effective cosmological constant $ \Lambda $.\\
The ordinary matter fields, in the form of a radiation fluid, and the spatial curvature K are usually neglected during inflation because their contribution to the energy density is redshifted away during the accelerated expansion.

\subsection{Slow-roll parameters}

We quantify now under which circumstances a scalar field may give rise to a period of inflation.
Considering the background scalar field $ \phi $ (homogeneous and isotropic) the equation of motion (\ref{eomForPhi}) becomes

\begin{equation}
	\label{eomHomogeneousField}
	\ddot{\phi} + 3H \dot{\phi} + V'(\phi) = 0.
\end{equation}

If the \textit{slow-roll} condition $ \phi^{2} \ll V(\phi) $ is satisfied, the scalar field slowly rolls down its potential. Such a period can be achieved if the inflaton field is in a region where the potential is sufficiently flat. Since the potential is flat we may also expect that $\ddot{\phi}$ is negligible as well. We will assume this is true and we will quantify this condition.\\
Requiring the slow-roll condition the Friedmann equation (\ref{friedmannEquations1}) becomes 

\begin{equation}
	\label{friedMannEqDuringInflation}
	H^{2} \simeq \frac{8\pi G}{3} V(\phi),
\end{equation}

where we assumed that the inflaton field dominates the energy density of the universe. Moreover, assuming also $\ddot{\phi}$ negligible we obtain the new equation of motion

\begin{equation}
	\label{newEQMphiDotDotNegligible}
	3H\dot{\phi} = -V'(\phi)
\end{equation}

Using (\ref{newEQMphiDotDotNegligible}) and the slow-roll condition (\ref{conditionFlatPotential}) we obtain

\begin{equation}
	\label{condition1}
	\frac{(V')^{2}}{V} \ll H^{2}
\end{equation}

and, considering $\ddot{\phi} \ll 3H\dot{\phi}$,

\begin{equation}
	\label{condition2}
V'' \ll H^{2}	.
\end{equation}

These two conditions represent the flatness conditions of the potential which are conveniently parametrized in terms of the so-called \textit{slow-roll parameters}.
The slow-roll parameters quantify the slow-roll regime dynamics in order to give predictions of specific models and to compare it with others and with observations.\\
Firstly, we define the $\epsilon$ parameter

\begin{equation}
	\label{epsilon}
	\epsilon = - \frac{\dot{H}}{H^{2}}
\end{equation}

that describes how much $ H $ changes during inflation.
To relate this variable to the slow-roll relation $ \dot{\phi}^{2} \ll V(\phi) $ let us derive the first Friedmann equation (neglecting the curvature)

\begin{equation}
	\label{derivingEpsilon1}
	H^{2}=\frac{8\pi G}{3}\rho_{\phi} = \frac{8\pi G}{3} \Big (\frac{1}{2}\dot{\phi}^{2} + V(\phi) \Big )
\end{equation}

obtaining,

\begin{equation}
	\label{FriedmannDerived}
	2H\dot{H} = \frac{8\pi G}{3} \Big (\dot{\phi} \ddot{\phi} + V'(\phi)\dot{\phi} \Big ).
\end{equation}
Now, inserting the equation of motion of the inflaton $ \ddot{\phi} = -3H\dot{\phi} - V'  $ in (\ref{FriedmannDerived}) we obtain

\begin{equation}
	\label{Hdot}
	\dot{H} = -4\pi G\dot{\phi}^{2}
\end{equation}

Finally, using $ H^{2}=(8\pi G/3)V(\phi) $  

\begin{equation}
	\label{epsilon2}
	\epsilon= - \frac{\dot{H}}{H^{2}}=4\pi G \frac{\dot{\phi}^{2}}{H^{2}} \simeq \frac{3}{2} \frac{\dot{\phi}^{2}}{V(\phi)}
\end{equation}

where the last equality si valid only in the slow-roll regime. Therefore, we can interpret $\epsilon$ as the ratio between the kinetic energy and the potential. \\
Hence assuming $ V(\phi) \gg \dot{\phi}^{2}  $ we obtain 

\begin{equation}
	\label{conditionEpsilon}
	\epsilon \ll 1 .
\end{equation}
Moreover, exploiting $ H\dot{\phi} \simeq -V' $, we can write

\begin{equation}
	\label{newEpsilon}
	\epsilon=4\pi G \frac{\dot{\phi}^{2}}{H^{2}} \simeq \frac{1}{16\pi G} \Big (\frac{V'}{V} \Big )^{2}	
\end{equation}

which means that if $\epsilon \ll 1 $ , $V'$ is small and the potential is flat. Thus, $\epsilon$ quantifies the flatness of the potential.\\

Considering the second slow-roll condition $ \ddot{\phi} \ll 3H\dot{\phi} $, we can define the second slow-roll parameter as 

\begin{equation}
	\label{etaParameter}
	\eta = - \frac{\ddot{\phi}}{H\dot{\phi}} \ll 1
\end{equation}

As we have done for $\epsilon$ we can relate this expression to the potential. To do this we can derive $ \dot{\phi} \simeq -V'/3H $, obtaining

\begin{equation}
	\label{phiDerivedEtaParameter}
	\ddot{\phi} = -\frac{V'' \dot{\phi}}{3H} + \frac{\dot{H}}{3H^{2}}V'.
\end{equation}
Plugging this into the definition of $\eta$ we obtain

\begin{equation}
	\label{eta2}
	\eta \simeq \frac{V''}{3H^{2}} - \frac{\dot{H}}{H^{2}}\frac{V'}{3H\dot{\phi}} \simeq \eta_{V} - \epsilon
\end{equation}

where $ \frac{V'}{3H\dot{\phi}} \simeq -1 $ and we have defined $ \eta_{V}=\frac{V''}{3H^{2}} $. Thus again, having $\eta \ll 1$ means to have a flat potential.\\

A successful period of inflation requires that $ \epsilon, |\eta| \ll 1 $. Moreover, exists a hierarchy of slow-roll parameters: for example, one can define the slow-roll parameter related to the third derivative of the potential

\begin{equation}
	\label{secondOrderParameter}
	\xi^{2} = \Big (\frac{1}{4\pi G} \Big )^{2} \Big (\frac{V'V'''}{V^{2}}\Big )
\end{equation}

which is a second order slow-roll parameter. The third derivative of the potential corresponds to an eventual self-interaction of the inflaton field. \\
One can use these parameters with the data collected from observations to reconstruct the shape of the potential.\\
At first-order in the slow-roll parameters $\epsilon$ and $\eta$ can be considered constant, since the potential is very flat and their derivatives are higher orders in these parameters. In fact it is easy to see that $ \dot{\epsilon},\dot{\eta} = \mathcal{O}(\epsilon^{2},\eta^{2}) $.\\

If we written $ \epsilon = - \dot{H}/H^{2} $ we can notice that

\begin{equation}
	\label{scaleFactorParameter}
	\ddot{a}=\dot{\dot{a}}=\dot{(aH)}=\dot{a}H + a\dot{H}=aH^{2}\Big (1+\frac{\dot{H}}{H^{2}}\Big ) = aH^{2}(1-\epsilon) .
\end{equation} 

Thus, inflation can be attained only if $\epsilon < 1$. As soon as this condition fails, inflation ends.\\
This condition alone can be sufficient to realise inflation. However, having also $\eta \ll 1 $  assure that inflation lasts for long enough. In fact, $ \eta = - \frac{\ddot{\phi}}{H\dot{\phi}}  \ll 1 $ both ensures that inflation is an attractor solution and that $\dot{\phi}$ remains constant and small for long enough. In other worlds, $\eta$ controls the duration of inflation.\\

Despite the semplicity of the inflationary theory, the number of inflationary models that have been proposed so far is enormous, differing for the kind of potential and for the underlying particle phyisics theory. In the second chapter we will discuss about the most important models, but we just to mention here that the main classification in connection with the observations is the one in which the single-field inflationary models are divided into three broad groups as \textquotedblleft small field", \textquotedblleft large field" (or chaotic) and \textquotedblleft hybrid" type, according to the region occupied in the ($\epsilon$ - $\eta$) space by a given inflationary potential.\\

\section{Inflation and cosmological perturbations}
The description of the universe as a perfectly homogeneous and isotropic FRW model is an idealitation. Actually, we are interested in deviations from homogeneity and isotropy that enable us to characterise different models.\\
So far we have considered only the dynamics of a homogenous scalar field driving inflation. But to investigate inflation models in more detail and to test theoretical predictions against cosmological observations we need to consider inhomogeneous perturbations. \\
Besides the background inflationary dynamics, we have the evolution of the quantum fluctuactions of the inflaton field $ \delta\phi(t,\textbf{x}) $. In the inflationary model there are primordial energy density perturbations, associated with these vacuum fluctuations, which survive after inflation and are the origin of all the structures in the universe.\\
Once the universe became matter dominated ($ z \simeq 3200 $) primeval density inhomogeneites ($ \delta \rho/\rho \simeq 10^{-5} $) were amplified by gravity and grew into the structure we see today. The existence of these inhomogeneities was in fact confirmed by the COBE discovery of CMB anisotropies.\\
In this section we summarise how the quantum fluctuations of a generic scalar field evolve during an inflationary stage. For more details see \cite{Liddle:intro},\cite{NonGauss:Intro},\cite{Dodelson:Chap1}.

\subsection{Quantum Fluctuations}
Consider for semplicity a scalar field $ \phi $ (the inflaton) with an effective potential $ V(\phi) $ in a pure de Sitter stage, during which the Hubble rate $ H $ is constant. 
We first split the field $ \phi(\tau,\textbf{x}) $ in the homogeoneous classical part, $ \phi(\tau) $, and its fluctuations $ \delta\phi(\tau,\textbf{x}) $

\begin{equation}
	\label{splitInflatonTau}
	\phi(\tau,\textbf{x})=\phi(\tau) + \delta \phi(\tau,\textbf{x})
\end{equation}

where $ \tau $ is the conformal time, related to the cosmic time $ t $ through $ d\tau=dt/a(t) $.\\
We consider the Fourier transform of the fluctuations
\begin{equation}
\label{fourierTransform}
\delta\phi(\tau,\textbf{x})=\frac{1}{(2\pi)^{3}}\int d^{3}k e^{i \textbf{k}\cdot\textbf{x}}\delta\phi(\tau,k).
\end{equation}
Redefining  the scalar field as 
\begin{equation}
\label{fieldRedefinition}
\widetilde{\delta \phi}	= a\delta\phi
\end{equation}
 we can promote it to an operator which can be decomposed as 
 \begin{equation}
 	\label{quantitation}
 	\widetilde{\delta \phi}(\tau,\textbf{x})=\int \frac{d^{3} \textbf{k}}{(2\pi)^{3/2}} \big[u_{k}(\tau)a_{\textbf{k}}e^{i \textbf{k}\cdot\textbf{x}} + u_{k}^{*}(\tau)a_{\textbf{k}}^{\dagger}e^{-i \textbf{k}\cdot\textbf{x}}\big]
 \end{equation}
 
 where we have introduced the creation and annihilation operators $ a_{\textbf{k}} $ and $ a^{\dagger}_{\textbf{k}} $.\\
 The creation and annihilation operators for $\widetilde{\phi}$ satisfy the standard commutation relations
 \begin{equation}
 	\label{commutationRelations}
 \big[a_{\textbf{k}},a_{\textbf{k}'}] = 0, \qquad \big[a_{\textbf{k}},a_{\textbf{k}'}^{\dagger}] = \delta^{(3)}(\textbf{k}-\textbf{k}')
 \end{equation}
and the modes $ u_{k}(\tau) $ are normalized so that they satisfy the condition

\begin{equation}
	\label{normalitation}
	u^{\ast}_{k} u_{k}' - u_{k} u'^{\ast}_{k} = -i,
\end{equation}
where primes denote derivatives with respect to the conformal time $ \tau $. \\
Expanding the equation of motion for the scalar field (\ref{eomForPhi}) in the fluctuations $\delta\phi(\tau,\textbf{x})$ in Fourier space we obtain
\begin{equation}
	\label{eomFluctuations}
	u''_{k}(\tau) + \Big[k^{2} - \frac{a''}{a} + \frac{\partial^{2} V}{\partial\phi^{2}}a^{2}\Big] u_{k}(\tau) = 0.
\end{equation}

 where $ m^{2}_{\phi} = \partial^{2} V / \partial\phi^{2} $ is the effective mass of the scalar field. This equation describes an harmonic oscillator with a frequency changing in time, due to the expansion of the universe.\\
 The modes $ u_{k}(\tau) $ at very short distances must reproduce the form for the ordinary flat space-time quantum field theory. Thus, well within the horizon, in the limit $ k/aH \rightarrow \infty $, the modes should approach plane waves of the form
 
 \begin{equation}
 	\label{planeWave}
 	u_{k}(\tau) \rightarrow \frac{1}{\sqrt{2k}}e^{-ik\tau}.
 \end{equation}

Let us consider a special case where the inflaton is massless in a pure de-Sitter universe ($m_{\phi} = 0, H=constant $). In this situation the equation (\ref{eomFluctuations}) becomes
\begin{equation}
\label{eomMasslessDesitter}
u''_{k}(\tau) + \Big[k^{2} - \frac{a''}{a}\Big]u_{k}(\tau) = 0.
\end{equation}

Using $a d\tau=dt$, and $ a \propto e^{Ht}$ in a de-Sitter stage 
\begin{equation}
	\label{horizon}
	\frac{a''}{a}= \frac{2}{\tau^{2}}=2a^{2}H^{2}=\frac{2}{r_{H}^{2}}
\end{equation}
where $ r_{H} $ is the comoving Hubble radius.\\
Thus, we can study (\ref{eomMasslessDesitter}) in two different regimes: the \textit{sub-horizon} regime where $ \lambda_{phys} \ll H^{-1} $, $ k^{2} \gg a^{2}H^{2} \simeq a''/a $, and the \textit{super-horizon} regime where $ \lambda_{phys} \gg H^{-1} $, $ k^{2} \ll a^{2}H^{2} \simeq a''/a $.\\
In the sub-horizon case the equation of motion reduces to the wave equation

\begin{equation}
\label{waveEquationFluctuation}
u''_{k} + k^{2}u_{k}=0\  \rightarrow\  u_{k}(\tau)= \frac{1}{\sqrt{2k}}e^{-ik\tau} 
\end{equation}

and the field,

\begin{equation}
	\label{fieldSubHorizon}
	 \delta\phi_{k}=u_{k}/a=\frac{1}{a}\frac{1}{\sqrt{2k}}e^{-ik\tau}
\end{equation}

from which we can notice that it has a decreasing amplitude $ |\delta\phi| = 1/a\sqrt{2k} $, which depends on the inverse of $ a $.
In the super-horizon regime we obtain the equation

\begin{equation}
	\label{superHorizon}
	u''_{k} + \frac{a''}{a}u_{k}=0.
\end{equation}

This equation is solved by

\begin{equation}
	\label{solutionSuperHorizon}
	u_{k}(\tau) = B(k)a(\tau) + A(k)a^{-2}(\tau)
\end{equation}
where A, B are integration constants in $ \tau $ which depends on $ k $. In term of the field we get
\begin{equation}
	\delta\phi_{k} = B(k) + A(k)a^{-3}(\tau) \simeq B(k) = constant,
\end{equation}
where we have neglected the decaying term which gets washed away by inflation.\\
We can fix the amplitude of the growing mode, $ B(k) $, by matching the (absolute value of) this solution to the plane wave solution (\ref{fieldSubHorizon}) when the fluctuation with wavenumber k leaves the horizon ($ k = aH $)
\begin{equation}
	|B(k)|= \frac{1}{a\sqrt{2k}} = \frac{H}{\sqrt{2k^{3}}},
\end{equation}
So that the quantum fluctuations of the original scalar field $ \phi $ on super-horizon scales are constant,
\begin{equation}
	\label{frozenField}
	|\delta \phi_{k}| = \frac{|u_{k}|}{a} = \frac{H}{\sqrt{2k^{3}}}.
\end{equation}

From this simple computation we can see that inflation is able to provide a mechanism to generate density perturbations (and gravitational waves). To understand what is going on, a key ingredient is the decreasing with time of the comoving Hubble horizon $ (aH)^{-1} $ during inflation. The wavelength of a quantum fluctuation in the inflaton field soon exceeds the Hubble radius. The quantum fluctuations arise on scales which are much smaller than the comoving Hubble radius, which is the scale beyond which causal processes cannot operate. On small scales we can use the usual flat space quantum field theory to describe the scalar field vacuum fluctuations.\\
However, the inflationary expansion \textit{stretches} the wavelength of these fluctuations to outside the horizon. The quantum fluctuations of the inflaton are amplified (and frozen) on super-horizon scale, resulting in a net number of scalar field particles.\\
On large scales the perturbations just follow a classical evolution. Since microscopic physics does not affect the evolution of fluctuations when their wavelengths are outside the horizon, the amplitude of these inhomogeneites are "frozen" and fixed at some nonzero value $ \delta\phi $ at the horizon crossing.\\
The amplitude of the fluctuations on super-horizon scales then remains almost unchanged for a very long time, whereas the wavelength grows exponentially. Thus, these frozen fluctuations of the inflaton are equivalent to the appearance of a classical field $ \delta\phi $ that does not vanish after having averaged over some macroscopic interval of time. Moreover, the same mechanism also generates a stochastic background of gravitational waves.\\
The quantum fluctuations of the inflaton generate also fluctuations in the space-time metric, giving rise to perturbations of the curvature $ \mathcal{R} $. On super-horizon scales, curvature fluctuations are frozen in and considered as classical.\\
When the wavelength of these perturbations reenters the horizon, in the radiation or matter dominated epoch, the curvature perturbations of the space-time give rise to matter (and temperature) perturbations $ \delta \rho $ via the Poisson equation.\\
These fluctuations will then start growing, giving rise to the structure we observe today.

\subsection{Power spectrum}

To characterise the properties  of a perturbation field we introduce the \textit{power spectrum}.\\
Consider a random field $ f(t,\textbf{x}) $ that can be expanded in Fourier space as 
\begin{equation}
\label{fourierTransform}
	f(t,\textbf{x}) = \int \frac{d^{3}\textbf{k}}{(2\pi)^{3/2}} e^{i\textbf{k}\cdot \textbf{x}} f_{\textbf{k}}(t).
\end{equation}

The power spectrum $ P_{f}(k) $ can be defined by means the relation
\begin{equation}
\label{powerSpectrum}
\big<f_{\textbf{k}_{1}}f^{*}_{\textbf{k}_{2}}\big> \equiv \frac{2\pi^{2}}{k^{3}}P_{f}(k) \delta^{(3)}(\textbf{k}_{1}-\textbf{k}_{2})
\end{equation}
where the angled brackets denote the ensemble average. The power spectrum measures the amplitude of the fluctuations at a given scale $k$. Indeed, if we consider the last definition the mean square value of $f(t,\textbf{x})$ in real space is
\begin{equation}
\label{meanSquare}	
\big<f^{2}(t,\textbf{x})\big>= \int \frac{dk}{k} P_{f}(k).
\end{equation}

To describe the slope of the power spectrum  we define the $ \textit{spectral index}$   $n_{f}(k)$
\begin{equation}
\label{spectralIndex}
n_{f}(k) - 1 \equiv \frac{d \ lnP_{f}}{d \ ln k}.
\end{equation}  

For the inflaton field quantum fluctuations $ |\delta \phi_{k}| = \frac{|u_{k}|}{a} $,

\begin{equation}
	\label{fluctuations}
	\big < \delta\phi_{\textbf{k}_{1}}, \delta \phi^{*}_{\textbf{k}_{2}}\big>=\frac{2\pi^{2}}{k^{3}}|\delta \phi_{\textbf{k}_{1}}| \delta^{(3)}(\textbf{k}_{1}-\textbf{k}_{2}).
\end{equation}

Therefore,
\begin{equation}
\label{spectrumFluctuation}
P_{\delta \phi}(k) = \frac{k^{3}}{2\pi^{2}}|\delta \phi_{k}|^{2},		
\end{equation}

with $ \delta \phi_{k} \equiv u_{k}/a $.	

\subsection{Exact solution}
Now we briefly recap how to solve exactly the equation of motion for the modes $ u_{k} $ (\ref{eomFluctuations}). This equation can be rewritten in the form of \textit{Bessel equation} 
\begin{equation}
	\label{bessel}
	u''_{k}(\tau) + \Big[k^{2} - \frac{\nu^{2}-1/4}{\tau^{2}}\Big] u_{k}(\tau)=0.
\end{equation}
In this form, it is equivalent to the Bessel equation
\begin{equation}
	z^{2}y''(z) + zy'(z) + (z^{2}-\nu^{2})y(z)=0.
\end{equation}				
whose solutions are known to be of the form 
\begin{equation}
	\label{solutionPerturbations}
	u_{k}(\tau) = \sqrt{-\tau}\big [c_{1}(k)H^{(1)}_{\nu}(-k\tau) + c_{2}(k)H_{\nu}^{(2)}(-k\tau)].
\end{equation}
where $ H^{(1)}_{\nu} $ and $ H^{(2)}_{\nu} $ are the Henkel functions of first and second kind, respectively.
The parameter $ \nu $ can be expressed in terms of the slow roll parameters.\\
In the case of a quasi De-Sitter universe and  (little) massive scalar field we have the relation $ 3/2 - \nu \simeq \eta_{V} - \epsilon$.
 The requirement  of a light mass is due to the fact that if $ m^{2}_{\phi} \ge H^{2}$,  $ \delta \phi_{k} $ remains in the vacuum state and fluctuations get suppressed. From now we omit the subscript \textquotedblleft $V$" in $\eta$.\\
If we impose  that in the ultraviolet regime $ k \gg aH $ $ (-k\tau \gg 1) $ the solution matches the plane-wave solution $ e^{-ik\tau}/\sqrt{2k} $ that we expect in flat space-time. Knowing the asymptotic beheaviour of the Hankel functions on sub-horizon scales

\begin{equation}
\label{Hankel1}
H^{(1)}_{\nu}(x \gg 1) \sim \sqrt{\frac{2}{\pi x}} e^{i(x-\frac{\pi}{2}\nu-\frac{\pi}{4})} ,
\quad
H^{(2)}_{\nu}(x \gg 1) \sim \sqrt{\frac{2}{\pi x}} e^{-i(x-\frac{\pi}{2}\nu-\frac{\pi}{4})}
\end{equation}

and on super-horizon scales,

\begin{equation}
\label{Hankel2}
H^{((1)}_{\nu}(x \ll 1) \sim \sqrt{\frac{2}{\pi}}e^{-i\frac{\pi}{2}}2^{\nu-\frac{3}{2}}\frac{\Gamma(\nu)}{\Gamma(3/2)}x^{-\nu}	
\end{equation}

we can set in (\ref{solutionPerturbations})  $ c_{2}(k)=0 $ and $ c_{1}(k)=\frac{\sqrt{\pi}}{2}e^{i(\nu + \frac{1}{2})\frac{\pi}{2}} $.\\
We finally obtain for the fluctuations $ |\delta \phi_{k}| $ 
\begin{equation}
	\label{solutionDeltaPhi}
	|\delta \phi_{k}|= \frac{H}{\sqrt{2 k^{3}}}\Big (\frac{k}{aH}\Big)^{\frac{3}{2}-\nu},
\end{equation}

yielding for the power spectrum (\ref{spectrumFluctuation})

\begin{equation}
	\label{PowerSpectrumperturbation}
	P_{\delta \phi}(k) = \Big (\frac{H}{2\pi}\Big)^{2}\Big (\frac{k}{aH}\Big)^{3-2\nu}
\end{equation}

where $ \nu $ is given by $ 3/2 - \nu \simeq \eta_{V} - \epsilon$.\\
In the power spectrum just computed there is an inconsistency. In the computation the scalar field is perturbed on a unperturbed spacetime. Thus, we should also include perturbations of the metric to have a correct result. To do so, we need to consider scalar perturbations of the metric and use gauge invariant quantities. But before doing that, we are going to consider the tensorial perturbations of the metric: the gravitational waves.

\subsection{Gravitational waves}
Inflation predicts the existence of a scale invariant spectrum of primordial gravitational waves, sourced by the same quantum fluctuations described in the previous sections. Gravitational waves are only weakly coupled to matter fields, and move freely through the universe from the moment they are produced.\\
The perturbations of the inflaton field will induce perturbations of the metric. This leads to a stochastic background of gravitational waves (GW)  which are represented by tensor perturbations of the metric.\\
A stochastic background of waves is a continuos set of waves, fully characterized  only by their global statistic properties. It consists of a signal coming from every direction in the sky. It is different from the signals coming from astrophysical sources (merging neutron stars,  binary black holes...), which come from a specific direction in the sky. \\
In this section we explain how inflation can generate this stochastic background. 

We start considering the perturbed spatially flat FLRW metric, where we neglect scalar and vector perturbations,
\begin{equation}
	ds^{2} = -dt^{2} + a^{2}(t)[\delta_{ij} + h_{ij}]dx_{i}dx_{j}
\end{equation}
with $ h_{ij} $, in the so called Transvere-Traceless gauge (TT gauge), are such that
\begin{equation}
	h_{ij}=h_{ji} \qquad h^{i}_{i}=0 \qquad h^{i}_{j|i} = 0.
\end{equation}
 
 At linear level Einstein's equations for $ h_{ij} $ are 
 \begin{equation}
 	\ddot{h}_{ij} + 3H\dot{h}_{ij} - \frac{\nabla^{2}h_{ij}}{a^{2}}= \Pi^{TT}_{ij}
 \end{equation}
where $\Pi^{TT}_{ij}$ is a tensor, with the same properties of $ h_{ij} $, which is a source term coming from possible anisotropic stress of the matter source. It is related  to the last term of the stress-energy tensor of a perfect fluid $ T_{\mu\nu} = (\rho + P)u_{\mu}u_{\nu} + Pg_{\mu\nu} + \Pi_{\mu\nu}$, called anisotropic stress tensor, which can get a contribution in the case of astrophysical sources when we have a non vanishing quadrupole moment. We will see that this term is also important to describe the gravitational waves emitted in the reheating phase.\\
However, at first order, it is vanishing during single field inflation and the equation of $ h_{ij} $ becomes 
\begin{equation}
	\label{eqh}
		\ddot{h}_{ij} + 3H\dot{h}_{ij} - \frac{\nabla^{2}h_{ij}}{a^{2}}=0,
\end{equation}
which is similar to the equation for the quantum vacuum fluctuation in the case of a massless scalar field.\\
Since there is no source term, GWs are the intrinsic quantum fluctuations of the metric. Moreover, they provide a \textit{smoking gun} of inflation and would be the first ever detected evidence of quantum gravity.\\
The equation (\ref{eqh}) describes the evolution of the tensor $ h_{ij} $, which has 2 independent DOF, corresponding to the two possible polaritations of GWs $ \lambda = (+,\times) $. Such object can be decomposed in Fourier space as 

\begin{equation}
	\label{fouriewrh}
	h_{ij}(\tau,\textbf{x}) = \sum_{+\times} \int \frac{d^{3} k}{(2\pi)^{3}}e^{i\textbf{k}\cdot \textbf{x}}h_{\lambda}(\textbf{k},\tau)\epsilon^{\lambda}_{ij}(\textbf{k})
\end{equation}
where $ \epsilon^{\lambda}_{ij}(\textbf{k}) $ are the polaritation tensors, which satisfy
\begin{equation}
	\epsilon_{ij}=\epsilon_{ji} \qquad \epsilon^{i}_{i}=0 \qquad  k^{i}\epsilon_{ij}(\textbf{k}) =0
\end{equation}
 with normalitation conditions
 \begin{equation}
 \epsilon_{ij}^{\lambda}(\textbf{k})\epsilon^{*ij}_{\lambda'}(\textbf{k})=\delta_{\lambda\lambda'}
 \qquad
 \big (\epsilon_{ij}^{\lambda}(\textbf{k})\big)^{*} = \epsilon_{ij}^{\lambda}(-\textbf{k}).
 \end{equation}
Considering a plane monochromatic gravitational wave propagating in the $\hat{z}$ direction, in Fourier space we have

\begin{equation}
	\epsilon^{+}_{ij} = 
	\begin{pmatrix}
		1 & 0 \\
		0 & -1
	\end{pmatrix}
\qquad
\epsilon^{\times}_{ij} = 
\begin{pmatrix}
	0 & 1 \\
	1 & 0
\end{pmatrix}
\end{equation}
\begin{equation}
	h_{ij}(\textbf{k},\tau) = h_{+}(\textbf{k},\tau)\epsilon_{ij}^{+}(\textbf{k}) + 
	h_{\times}(\textbf{k},\tau)\epsilon_{ij}^{\times}(\textbf{k})
\end{equation}

and, the tensor $ h_{ij} $ satisfies

\begin{equation}
	\ddot{h}_{\lambda} + 3H\dot{h}_{\lambda}+ k^{2}\frac{h_{\lambda}}{a^{2}}=0
\end{equation}
which is the same for each polaritation state.\\
On super-horizon scales, $ k\ll aH $, the solution for $ h_{+,\times} $ is given by a constant plus a decaying mode. Using the canonical normalitation
\begin{equation}
|h_{+,\times}|= \sqrt{32\pi G}|\phi_{+,\times}| = \sqrt{32\pi G} \frac{H}{\sqrt{2k^{3}}}\Big (\frac{k}{aH}\Big)^{-\epsilon}.
\end{equation}
On sub horizon scales ($ k \gg aH $), $ h_{+,\times}=\frac{e^{-ik\tau}}{a(\tau)} $.\\
For the power spectrum we obtain
\begin{equation}
\label{spectumGW}
P_{T}(k)=\frac{k^{3}}{2\pi^{2}}<h^{*}_{ij}h^{ij}>=\frac{16}{M_{pl}^{2}}\Big(\frac{H}{2\pi}\Big )^{2}\Big (\frac{k}{aH}\Big)^{-2\epsilon}
\end{equation}   
where $ H $ indicates the Hubble rate during inflation and we have summed the two polaritations $ (+,\times) $.\\
Therefore we can define the spectral index of inflationary gravitational waves as
\begin{equation}
n_{T} = \frac{d\ ln P_{T}}{d\ ln k} = -2\epsilon.
\end{equation}
In the simplest models one has $ \epsilon > 0 $ so $ n_{T} $ is always red- tilted (on smaller scales the amplitude decreases).
Since during inflation $P_{T} $ $\sim$ $ H^{2} $ and $ H^{2} \simeq \frac{V}{M^{2}_{pl}} $, detecting the tensor spectrum would give us the energy scale of inflation ($ E_{inf} \simeq V^{1/4} $, see later). 

\subsection{Primordial curvature perturbation}
In the standard slow-roll inflationary models the fluctuations of the inflaton field are responsible for the curvature perturbations.
 As said, they are (nearly) frozen on super-horizon scales. When they reenters the horizon lead to pertubations of matter that give rise the structure we see today.\\
To characterise the scalar and curvature perturbations we need gauge invariant quantities.
A complete treatment of this argument is in \cite{NonGauss:Intro}. Here we just summarise the main points.


Consider the perturbed FRW metric at first order including only scalar perturbation and expressed with the conformal time $ \tau=\int dt/a(t) $
\begin{equation}
\label{metricPerturbed}
ds^{2}=a^{2}(\tau) \big [-(1+2\Psi)d\tau^{2} + (1-2\Phi)\delta_{ij}dx_{i}dx_{j} \big]
\end{equation}

The first gauge invariant quantity we consider is
\begin{equation}
	\label{zeta}
	-\zeta \equiv \hat{\Phi} + \mathcal{H}\frac{\delta \rho}{\rho'}
\end{equation}
where $\mathcal{H} \equiv a'/a$ is the Hubble parameter in conformal time and the prime denote differentiation w.r.t it. $\hat{\Phi}$ is referred to as the \textit{curvature perturbation}. This quantity, however, is not gauge invariant since it changes under a transformation on costant time hyper-surfaces $\tau$ $\rightarrow$ $ \tau + \alpha $. Instead, combining the $\hat{\Phi}$ transformation and the gauge transformation for scalars comes out that $\zeta$ in (\ref{zeta}) is gauge invariant. This quantity is called \textit{gauge-invariant curvature perturbation of the uniform energy-density hypersurfaces}.\\
To obtain the $\zeta$ power spectrum consider another gauge invariant quantity called \textit{curvature perturbation on comoving hyper-surfaces}. In the case of a stress-energy tensor of a single scalar field it reads
\begin{equation}
	\label{R}
	\mathcal{R} \equiv \hat{\Phi} + \frac{\mathcal{H}}{\phi'}\delta \phi
\end{equation}

The comoving curvature perturbation $\mathcal{R}$ is related to the curvature perturbation $\zeta$ by
\begin{equation}
	- \zeta = \mathcal{R} + \frac{2\rho}{9(\rho + P)} \Big (\frac{k}{aH}\Big)^{2}\Psi,
\end{equation}
where $\psi$ is the perturbation that appears in the metric. 
From this relation we obtain that on large scales $\mathcal{R} \simeq -\zeta$.\\
In the previous sections we obtained the power spectrum of  the primordial fluctuations of the inflaton (\ref{PowerSpectrumperturbation}). However, we computed it without taking into account the perturbation of the metric.\\
To do so, we define a new gauge-invariant quantity called  \textit{Sasaki-Mukhanov variable}
\begin{equation}
	\label{sasaki-mukhanov}
	\mathcal{Q}_{\phi} \equiv \delta \phi + \frac{\phi'}{\mathcal{H}}\Phi 
\end{equation}
Introducing the field $\tilde{\mathcal{Q}}_{\phi}=aQ_{\phi}$, the Klein-Gordon equation reads \cite{NonGauss:Intro}
\begin{equation}
	\label{finalEOM}
\tilde{\mathcal{Q}}'' + \Big (k^{2} - \frac{a''}{a} + \mathcal{M}_{\phi}^{2}a^{2}\Big)\tilde{\mathcal{Q}}_{\phi} = 0
\end{equation} 
where 
\begin{equation}
 \mathcal{M}^{2}_{\phi}=\frac{\partial^{2} V}{\partial \phi^{2}} - \frac{8\pi G}{a^{3}}\Big (\frac{a^{3}}{H}\dot{\phi}^{2}\Big)
\end{equation}
is an effective mass of the inflaton field. At lowest orders in the slow-roll parameters the latter expression reduces to $ \mathcal{M}^{2}_{\phi}/H^{2} = 3\eta-6\epsilon $. Solving (\ref{finalEOM}) by means of the Hankel functions, as we did in the previous sections, we obtain at super-horizon scales and at  lowest order in the slow-roll parameters the complete solution
\begin{equation}
	\label{solutionPerturbation}
	|Q_{\phi}(k)| = \frac{H}{\sqrt{2k^{3}}}\Big (\frac{k}{aH}\Big)^{3/2-\nu}
\end{equation}
where $ \nu \simeq 3/2 + 3\epsilon - \eta $.\\
This solution leads a power spectrum
\begin{equation}
\label{PSQ}
P_{\mathcal{Q}}=\Big (\frac{H}{2\pi} \Big)^{2}\Big(\frac{k}{aH}\Big)^{3-2\nu}.
\end{equation}
Now, returning to the gauge-invariant curvature perturbation $ \mathcal{R} $ (\ref{R}), we can easily express it in function of the Sasaki-Mukhanov variable. Using (\ref{sasaki-mukhanov}) results
\begin{equation}
	\label{expressionR}
	\mathcal{R}=\dfrac{\mathcal{H}Q_{\phi}}{\phi'} = \frac{H Q_{\phi}}{\dot{\phi}}
\end{equation}
where we have express $ \mathcal{R} $ in terms of the cosmic time.
Finally, using (\ref{PSQ}), we obtain the power spectrum of the curvature perturbation $\mathcal{R}$:
\begin{equation}
\label{PR}
P_{\mathcal{R}}=\Big (\frac{H}{\dot{\phi}}\Big)^{2}P_{\mathcal{Q}}= \Big (\frac{H^{2}}{2\pi \dot{\phi}}\Big)^{2}\Big(\frac{k}{aH}\Big)^{3-2\nu} \simeq \Big(\frac{H^{2}}{2\pi\dot{\phi}}\Big)^{2}_{*}
\end{equation}
where  the asterisk denotes quantities evaluated at the epoch a given perturbation mode leaves the horizon during inflation, that is $ k=aH $.\\
The last equation shows that the curvature perturbations remains  time-independent on super-horizon scales. In the \textit{uniform curvature gauge}, where $ \Phi=0 $, we have $ \zeta \simeq - \mathcal{H}\delta \rho/\rho' $. So, we can connect the inflaton perturbations to observable quantities.
The solution obtained for $\zeta$ is valid  throughout the differen evolution eras of the Universe until the mode remains super horizon.\\
From (\ref{PR}) we can easily obtain the spectral index of the curvature perturbation (at lowest order in the slow-roll approximation)
\begin{equation}
\label{spectralIndex}
n_{\mathcal{R}}-1\equiv \frac{d\ ln P_{\mathcal{R}}}{d\ ln k} = 3 - 2\nu=-6\epsilon + 2\eta.
\end{equation}
Inflationary models predict a power spectrum od density perturbations very close to 1. The specific case in which $ n_{s}=1 $ is called \textit{Harrison-Zel'dovich} spectrum that means that the amplitude of the inflaton pertubations does not depend on the cosmological scale.\\ 
The curvature mode is the quantity which allows to connect the primordial perturbations produced during inflation to the observables.
This result comes from the fact that in single-field slow roll models the intrinsic entropy perturbation of the inflaton field is negligible on large scales. This result holds also during the reheating phase after inflation\cite{NonGauss:Intro}. 

\subsection{Consistency relation}
In the single-field models a important consistency relation holds. To derive it, we introduce the \textit{tensor-to-scalar ratio}
\begin{equation}
\label{tensorScalarRatio}
r(k_{*})\equiv \frac{P_{T}(k_{*})}{P_{S}(k_{*})}
\end{equation}
that yields  the amplitude of the GW with respect to that of the scalar perturbations at some pivot scale $ k_{*} $.\\
We can rewrite the power spectrum $ P_{\mathcal{R}} $ in (\ref{PR}) as function of the slow-roll parameters using the fact that $ \epsilon=-\dot{H}/H^{2}=4\pi G\dot{\phi^{2}}/H^{2} $:
\begin{equation}
	\label{PSRelatedToEpsilon}
P_{\mathcal{R}}(k)=\frac{1}{2M_{P}^{2}\epsilon}\Big (\frac{H}{2\pi}\Big)^{2}\Big (\frac{k}{aH}\Big)^{n_{\mathcal{R}}-1}	
\end{equation}
that yields $ r=16\epsilon $.\\
Furthermore, we have shown that a nearly scale-invariant spectrum of tensor modes is expected, being $ n_{T}=-2\epsilon $. Therefore at the lowest order in the slow-roll parameters, one finds the consistency relation
\begin{equation}
	\label{consistencyRelation}
	r=-8n_{T}
\end{equation}
This equality can be proved only with a measure of the tensor power spectrum (not only the amplitude, but also  its spectral index \textit{i.e.} its shape).\\
Since a large spectral index would invalidate the consistency relation, it will be very hard to measure any scale dependance of the tensors assuming the consistency relation valid.\\
At present we have only an upper bound on the tensor-to-scalar ratio from the joint analysis of BICEP2, Keck Array data and Planck: $ r_{0.05}<0.07 $ at  95$\% $ C.L. \cite{Bicep2:Intro}, assuming valid the consistency relation. The subscript indicates the pivot scale in $ Mpc^{-1} $ units.\\

Finally we can connect the energy scale of inflation to the tensor-to-scalar ration $ r $.\\
From $ H^{2}=8\pi GV/3 = V/3M_{pl}^{2}$ we can link the energy scale of inflation, at the time when the pivot scale leaves the horizon, directly to the parameter $\epsilon$ using (\ref{PSRelatedToEpsilon})
\begin{equation}
	V=24\pi^{2}M_{pl}^{4}P_{\mathcal{R}}\epsilon =(3\pi^{2}P_{\mathcal{R}}/2)M_{pl}^{4}r.
\end{equation}
Thus, considering  the scalar amplitude estimated  by the Planck Collaboration \cite{Plank2015:Chap1} one gets  the following relation between the energy scale of inflation at the time when the pivot scale leaves the Hubble radius, and the tensor-to-scalar ratio:
\begin{equation}
	\label{energyScaleInflation-r}
	V=(1.88\times10^{16}GeV)^{4}\frac{r}{0.10}.
\end{equation}
Then we have a direct link between $ r $ and the energy scale of inflation.\\

In the next chapter we'll overview the most important models of inflation and we'll introduce the important stage of reheating by means a pertubative toy model.\\
The reheating phase is the main focus of this work since it leads very intersting effects and physics. Moreover, the gravitational waves generated by reheating are very different from those generated by inflation and could be detected by future experiments providing a observational window of such period.

\chapter{Inflation zoology and Reheating}
So far we have not discussed the form of the inflaton potential, $ V(\phi) $. A simple model of inflation was proposed by Guth in 1981 to solve the horizon and flatness problem discussed in the first chapter\\REFERENCE.
In this model, called "old inflation" scenario, the inflaton was trapped in a metastable false vacuum and had to exit to the true vacuum via a first-order transition. Thus, in this scenario inflation is as exponential expansion of the universe in a supercooled false vacuum state that makes the universe very big and very flat. However, as pointed out by Guth in the paper, this type of inflation produce a random nucleation of bubbles that lead a highly inhomogeneous universe.\\
This problem is solved by the "new inflation" [REFERENCE]. In this model, inflation may begin either in the false vacuum, or in an unstable state at the top of the effective potential. The key difference between the new inflationary scenario and the old one is that the inflation does not occur in the false vacuum state. Unfortunately, also this picture had problems. It works only if the effective potential of the field $ \phi $ has a very flat plateau near $\phi=0$, which is somewhat artificial. Moreover, in most versions of this scenario the inflaton field  has an extremelly small coupling constant, so it could not be in thermal equilibrium  with other matter fields.\\
In the beginning of the 80's, on the basis of all available observations(CMB, abundance of light elements), everybody believed that the universe was in a state of thermal equilibrium from the very beginning and the stage of inflation was just an intermediate stage of the evolution of the universe. Thus, also new inflation represented an incomplete modification of the big bang theory.\\
In the 1983 the chaotic inflation resolved all problems of old and new inflation[REFERENCE]. According to this scenario, inflation may begin even if there was no thermal equilibrium in the early universe and it may occur in the scenarios with simplest quadratic potential. Moreover, it is not limited to theories with polynomial potentials: chaotic inflation occurs in any theory where the potential has a sufficiently flat region, which allows the existence of the slow-roll regime, as described in the previous chapter. \\
The different kinds of single-field inflationary models can classified in the following way. The first class consists of the \textit{large field} models, in which the initial value of the inflaton is large and it slow rolls down toward  the potential minimum at smaller $\phi$. Chaotic inflation is one of the representative models of this class. The second class consists of the \textit{small fied} models, in which the inflaton field is small initially and slowly evolves toward the potential minimum at larger $\phi$. An example of this type is new inflation. The third class consists of the hybrid inflation models, in which inflation typically ends by a phase transition triggered by the presence of a second scalar field.\\
Several models of inflation can involve a coupling with gauge fields or other scalar fields. These models are very interesting because they can give rise to a source of gravitational waves and curvature perturbations. \\
The main reheating models discussed in the literature are concentrated on the end of inflation in the chaotic and hybrid scenarios. Thus, we start this chapter reviewing these important models. After that, the observable quantities on the CMB and the signatures of the gravitational waves produced in these models are studied. In the end of the chapter we'll start to focus on the main topic of this work: the reheating phase. We'll start considering an elementary, perturbative scenario of reheating. From the next chapter we'll see that, instead, reheating have very complicate and non-linear dynamics. 

\section{Inflationary models}
\subsection{Guth's Inflation}
We start reviewing the Guth model of inflation because it contains very interesting points such as entropy production and the random nucleation of bubbles, phenomena that also occurs in non-linear stages of reheating.\\
This model was introduced to solve the horizon and flatness problem: initially, in fact, the early universe was assumed to be highly homogeneous, despite to the fact that separated regions were causally disconnected, and the initial value of the Hubble constant had to be extremelly fine-tuned to produce the flat universe we see today.\\
In this scenario the universe is assumed to be homogeneous and isotropic and then is described by the Robertson-Walker metric, that we rewrite:

\begin{equation}
	\label{metricCh2}	
	ds^{2}   = - dt^{2} + a^{2}(t)\Big[\frac{dr^{2}}{1-Kr^{2}}  +  r^{2}(d\theta^{2} + sin^{2} \theta d\phi^{2})\Big] . 
\end{equation}
where $ k=+1,-1,0 $ denotes a closed, open, or flat universe; $ a(t) $ is the scale factor.
As seen, the evolution of the scale factor is governed by the Friedmann equations

\begin{equation}
	\label{friedmannEquations1Chap2}
	H^{2}=\frac{8\pi G}{3}\rho - \frac{K}{a^{2}},
\end{equation}
\begin{equation}
	\label{friedmannEquations2Chap2}	
	\frac{\ddot{a}}{a} = -\frac{4\pi G}{3}(\rho + 3P),
\end{equation}

where $ H  $ is the Hubble constant. We rewrite the conservation of energy as

\begin{equation}
\label{Chap2:ConservationEnergy}
\frac{d}{dt}(\rho a^{3})=-p\frac{d}{dt}(a^{3}).
\end{equation}
In the standard big-bang model one also assumes that the expansion is adiabatic,

\begin{equation}
\label{Chap2:entropy}
\frac{d}{dt}(sa^{3})=0,
\end{equation}
where $ s $ is the entropy density.\\
To study the evolution of the universe we need an equation of state for matter. At high temperatures, however, is a good approximation to consider an ideal quantum gas of massless particles. Let $ g_{*}(T) $ the effective number of degrees of freedom, the thermodynamics functions are given by

\begin{equation}
\label{Chap2EnergyDensity}
\rho=\frac{\pi^{2}}{30}	g_{*}(T)T^{4,}	
\end{equation}
\begin{equation}
\label{s}
s=\frac{2\pi^{2}}{45}g_{*}(T)T^{3},
\end{equation}
\begin{equation}
	\label{n}
	n=\frac{\zeta(3)}{\pi^{2}}g'(T)T^{3}
\end{equation}

where $ n $ denotes the particle number density and $\zeta(3)=1.202...$ is the Riemann zeta function.











\chapter{Preheating and Gravitational Waves Production}

\chapter{Other Models of Preheating}

\chapter{Non Linear Evolution After Preheating: Bubbly Stage}

\chapter{Thermalitation}

\chapter{Signatures of Reheating}

\begin{thebibliography}{2}
	
	\bibitem{Liddle:intro} A. R. Liddle and D. H. Lyth, \emph{Cosmological inflation and large-scale structure} (New York: Cambridge University Press, 2006).
	
	\bibitem{NonGauss:Intro} N. Bartolo, E. Komatsu, S. Matarrese and A. Riotto, Phys.Rept. \textbf{402} (2004) 103, astro-ph/0406398.
	
	\bibitem{GWFromInflation:Intro} M. C. Guzzetti, N. Bartolo, M. Liguori and S. Matarrese (2016) arXiv:1605.01615 [astro-ph.CO]. 
	
	\bibitem{Guth:Intro} A. Guth, Phys. Rev. D 23 (1981) 347.
	
	\bibitem{COBE1:intro} K. M. G´orski, et al. Astrophys. J. 464 (1996) L11 astro-ph/9701191.
	
	\bibitem{COBE2:intro} G. F. Smoot, et al., Astrophys. J. 396 (1992) L1.

    \bibitem{WMAP:intro} H. V. Peiris, et al., Astrophys. J. Suppl. 148 (2003) 213.
	
	\bibitem{Planck2018:intro} Y. Akrami et al., Astron. Astrophys. 641, A9 (2020).
	
	\bibitem{Steigman:nucleosynthesisIntro} G. Steigman, Ann.Rev.Nucl.Part.Sci. \textbf{57} (2007) 463-491, 0712.1100.
	
	\bibitem{ReheatingPredictionsSingleFieldModel:intro} J.L. Cook et al., JCAP 1504 (2015) 047, 1502.04673.
	
	\bibitem{Bicep2:Intro} BICEP2, Keck Array, P.A.R. Ade et al., Phys. Rev. Lett. 116 (2016) 031302, 1510.09217.
	
	\bibitem{Bicep2BMode:Intro} \textbf{BICEP2 Collaboration} Collaboration, P. Ade et al., Phys. Rev. Lett. \textbf{112} (2014) 241101, 1403.3985.
	
	\bibitem{COre:intro} COrE, F.R. Bouchet et al., (2011), 1102.2181.
	
	\bibitem{PRISM:intro} PRISM, P. Andrè et al., JCAP 1402 (2014) 006, 1310.1554.
	
	\bibitem{LIGO:intro} LIGO Scientific, J.Aasi et al., Class. Quant. Grav. 32 (2015) 074001, 1411.4547.
	
	\bibitem{Lisa:Intro} A. Klein et al., Phys. Rev. D93 (2016) 024003, 1511.05581.
	
	\bibitem{InflationDynamicsAndReheating:chap1} B. A. Bassett, S. Tsujikawa and D. Wands, Rev. Mod. Phys. \textbf{78}, 537 (2006), astro-ph/0507632.
	
	\bibitem{Dodelson:Chap1} S. Dodelson, F. Schmidt, \emph{Modern Cosmology} (Academic Press, 2020).
	
	\bibitem{TopDefects:Linde} A.Linde, \textit{Particle Physics and Inflationary Cosmology}, Harwood, Chur (1990), hep-th/0503203.
	
	\bibitem{Plank2015:Chap1} Planck, P.A.R. Ade et al., (2015), 1502.02114.
		
\end{thebibliography}	
	
\end{document}
